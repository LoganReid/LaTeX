\documentclass[letterpaper, 11pt]{article}
\usepackage{fullpage}
\usepackage[english]{babel}
\usepackage[utf8]{inputenc}
\usepackage{amsmath}
\usepackage{graphicx}
\usepackage[colorinlistoftodos]{todonotes}
\usepackage{array}
\usepackage{longtable}

\begin{document}
\begin{equation}
V = IR
\end{equation}
This equation equates voltage, V, to current, I, and resistance, R. This should be used in circuit analysis to find an unknown quantity of voltage, resistance, or current if you know the other two.
\begin{equation}
F = ma
\end{equation}
This equation equates force, F, to mass, m, and acceleration, a. This can be used in many situations, but is usually used to solve for the motion of an object under some force.
\begin{equation}
P = IV
\end{equation}
This equations computes power loss, P, from voltage, V, and current, I. This equation is used when you know the voltage and current through a component and want to know how much power it is dissipating.
\begin{equation}
t = 1/f
\end{equation}
This equation allows you to find the period, t, of a repeating signal if you know the frequency, f. This is used when given either period or frequency and need to use the other value in another equation.
\begin{equation}
X_C = 1/(2*pi*f*C)
\end{equation}
This equation allows you to calculate the reactance in ohms of a capacitor if you know the frequency, f, and capacitance, C. This is used in AC circuit analysis to find the reactance of a capacitor so you can calculate current or voltage through that component.
\end{document}
