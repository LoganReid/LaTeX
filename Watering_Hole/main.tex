%%%%%%%%%%%%%%%%%%%%%%%%%%%%%%%%%%%%%%%%%%%%%%%%%%%%%%%%%%%%%%%%%%%%%%%%%%%%%%%%
%2345678901234567890123456789012345678901234567890123456789012345678901234567890
%        1         2         3         4         5         6         7         8

\documentclass[letterpaper, 10 pt, conference]{ieeeconf}  % Comment this line out
                                                          % if you need a4paper
%\documentclass[a4paper, 10pt, conference]{ieeeconf}      % Use this line for a4
                                                          % paper

\IEEEoverridecommandlockouts                              % This command is only
                                                          % needed if you want to
                                                          % use the \thanks command
\overrideIEEEmargins
% See the \addtolength command later in the file to balance the column lengths
% on the last page of the document



% The following packages can be found on http:\\www.ctan.org
%\usepackage{graphics} % for pdf, bitmapped graphics files
%\usepackage{epsfig} % for postscript graphics files
%\usepackage{mathptmx} % assumes new font selection scheme installed
%\usepackage{times} % assumes new font selection scheme installed
%\usepackage{amsmath} % assumes amsmath package installed
%\usepackage{amssymb}  % assumes amsmath package installed

\title{\LARGE \bf
The Watering Hole: Innovations in Learning
}

%\author{ \parbox{3 in}{\centering Huibert Kwakernaak*
%         \thanks{*Use the $\backslash$thanks command to put information here}\\
%         Faculty of Electrical Engineering, Mathematics and Computer Science\\
%         University of Twente\\
%         7500 AE Enschede, The Netherlands\\
%         {\tt\small h.kwakernaak@autsubmit.com}}
%         \hspace*{ 0.5 in}
%         \parbox{3 in}{ \centering Pradeep Misra**
%         \thanks{**The footnote marks may be inserted manually}\\
%        Department of Electrical Engineering \\
%         Wright State University\\
%         Dayton, OH 45435, USA\\
%         {\tt\small pmisra@cs.wright.edu}}
%}

\author{Logan Reid% <-this % stops a space
\thanks{*This work was not supported by any organization}% <-this % stops a space
\thanks{Logan Reid is a student in EE 393 at the University of Washington
        {\tt\small }}%
}


\begin{document}



\maketitle
\thispagestyle{empty}
\pagestyle{empty}


%%%%%%%%%%%%%%%%%%%%%%%%%%%%%%%%%%%%%%%%%%%%%%%%%%%%%%%%%%%%%%%%%%%%%%%%%%%%%%%%
\begin{abstract}

Children who grow up in poverty are deprived of the process of learning not to take things for granted. In order to teach this principle to children, we developed a water bottle with a permeable bottom called The Watering Hole. The emotional response children experience when using The Watering Hole indicated that it facilitated their education flawlessly.

\end{abstract}


%%%%%%%%%%%%%%%%%%%%%%%%%%%%%%%%%%%%%%%%%%%%%%%%%%%%%%%%%%%%%%%%%%%%%%%%%%%%%%%%
\section{INTRODUCTION}

Currently in sub-Saharan Africa there is an enormous issue plaguing the population. This issue is rooted deep within the learning mechanisms experience only by those in extreme poverty. The fundamental issue is that impoverished people are simply unable to learn how to not take things for granted [1]. The healthy upbringing of children in developed nations involves the process of experiencing ownership, which transforms into complacency. Once the ownership is inevitably revoked, the child learns about the importance of not underestimating the value of something due to their own negative experiences. Using our understanding of this learning process, we developed the Watering Hole to educate children quickly and in a cost effective method.

\section{METHOD}

The key to learning that one should not take things for granted is the emotional response to the lack of an expected result [2]. We needed to find a way to allow these children to experience an underestimation of the value of something then take it away, thus facilitating their education. The key was to develop a device that would quickly inspire complacency only to remove it and generate the negative emotions indicative of learning.

In order to let these children experience this avenue of learning, we have developed a device known as The Watering Hole. The Watering Hole is designed to be a high density polyethylene [3] water bottle with a key piece of technology: a hole in the bottom. The idea behind this solution is that children will be excited to have a water bottle and to fill it with their scarce amount of water, only to have the water go straight through the bottle onto the ground. In theory, this will teach children about the importance of not taking things for granted by allowing them to experience ownership and loss all in one swift action.

\section{RESULTS}

We tested The Watering Hole by going to fifty impoverished sub-Saharan African villages and distributing Watering Hole devices to half the children in the tribe. We then observed the children when they gathered water from their village pump and compared their reactions to the expected reaction based on children who grew up in middle class America and to the reactions of children without The Watering Hole.

The children who used Watering Hole technology were initially excited about being able to transport their water, however then became surprised and cried about their lost water. This correlated with the expected reaction of betrayal and sadness that are indicative of this learning mechanism, based on surveys of middle-class American children. The children without Watering Hole technology simply poured the water into their hands and drank it without any emotional reaction. The data is illustrated in table 1.

\begin{table}[h]
\caption{results of experiment}
\label{table_example}
\begin{center}
\begin{tabular}{|c||c|}
\hline
Group & Percent describing experience as betrayal, etc.\\
\hline
Villagers with Watering Hole & 98\\
\hline
Villagers without Watering Hole & 0\\
\hline
Middle-class American & 95\\
\hline
\end{tabular}
\end{center}
\end{table}


\section{DISCUSSION}

Using the emotional response metric, the children who used The Watering Hole were able to go through the learning process typical of non-impoverished children. The children who didn’t use The Watering Hole did not experience any emotional changes, and thus did not learn the intended lesson. This indicates that The Watering Hole is successfully the first learning tool that can teach impoverished children the important lesson of avoiding complacency. 

The major limitation of this study was that there was no other metric than emotional response, which is typically less reliable than more measurable quantities. However, due to the sample size and consistency of results, we can conclude that this error source has not invalidated our findings. We can conclude that The Watering Hole facilitates the learning process behind not taking things for granted.

\section{CONCLUSION}

The Watering Hole has effectively solved the issue of impoverished children never learning how to not take things for granted in a cost effective and fast way. Our device provided a learning avenue that is typically shut off for extremely poor children by removing something that the child felt entitled to. The results of our study that utilized The Watering Hole showed that children that used our device experienced the same emotional response that typical middle-class children experience, which is indicative of learning this valuable lesson. Overall, The Watering Hole proved to be a cost effective and very fast method for impoverished children to learn how important it is to not take things for granted.
\nocite{*}

\bibliographystyle{IEEEtran}
\bibliography{IEEEabrv,annot.bib}

\end{document}
